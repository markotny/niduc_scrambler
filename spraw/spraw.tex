\documentclass[polish, 11pt]{article}
    
\usepackage[a4paper, margin=25mm]{geometry}
\usepackage{babel,polski}
\usepackage[utf8]{inputenc}
\usepackage[T1]{fontenc}
\usepackage{booktabs,multirow,multicol}

\usepackage{graphicx}
\graphicspath{{DVB/} {V34/} {img/}}

\usepackage{xcolor}
\usepackage[font=small,labelfont=bf]{caption}
\captionsetup[figure]{name=Rys.}

\newcommand{\results}[3][1.0]{
    \includegraphics[width=#1\textwidth]{#2}
    \captionof{figure}{#3\label{fig:#1}}
}

\begin{document}
\begin{titlepage}
    \centering
    {\scshape\LARGE Niezawodność i diagnostyka układów cyfrowych 2\\ projekt \par}
    \vspace{1cm}
    {\scshape\Large,,Unikanie utraty synchronizacji przy pomocy randomizacji (Scrambling)''\par}
    \vspace{2cm}
    {\itshape\Large Janusz Długosz --- 235746\/\par}
    {\itshape\Large Jakub Dorda --- 235013\/\par}
    {\itshape\Large Marcin Kotas --- 235098\/\par}
    \vfill
    Prowadzący:\par
    Doc.~dr inż.~Jacek \textsc{Jarnicki}

    \vfill

    {\large Wrocław, \today\par}

\end{titlepage}

\section{Wstęp}

\section{Scramblery - teoria}
    \subsection{Przeznaczenie}
	    Scramblery są kluczowym elementem warstwy fizycznej systemu transmisji bezprzewodowej, poza zapewnieniem
	    kodowania interwałowego oraz modulacji transmisji, powodami dla których używa się układów scaramblujących są:
	    \begin{itemize}
	    	\item Dokładne odtworzenie czasu w układzie odbiorczym potrzebnego do synchronizacji transmisji,
	    	bez konieczności używania redundantnych znaków końca ramki. Rozwiązuje to problem potrzeby stosowania
	    	układu odtwarzającego taktowanie. Automatyzuje kontrolę wzmocnienia sygnału oraz pracę innych układów
	    	przetwarzających sygnał, dzięki eliminacji długich ciągów zer i jedynek.
	    	\item Zmniejszenie zakłóceń sygnału między falami nośnymi, bardziej zróżnicowane dane zapewniają
	    	rozproszenie widma sygnału co przekłada się na mniejszą wrażliwość na zakłócenia między sąsiednimi
	    	kanałami transmisji.
	    	\item Prosta metoda szyfrowania transmisji, stosowana dla sygnałów analogowych. Zarówno scrambler
	    	w urządzeniu nadawczym jak i descrambler w urządzeniu odbiorczym muszą posiadać ten sam klucz/ciąg
	    	znaków w celu poprawnego dekodowania danych.
	    \end{itemize}
	    
    \subsection{Typy scramblerów}
	    \subparagraph{Addytywny (synchroniczny)\\}
		    Scramblery synchroniczne przetwarzają dane wejściowe przez zastosowanie podwójnej operacji xor z dwoma najmłodszymi
		    bitami pseudo losowego ciągu znaków oraz bitem wejścia. Po każdej operacji ciąg - klucz jest przesuwany o jedną
		    pozycję w prawo, a wynik xor dwóch najmłodszych bitów zapisywany na pozycji pierwszej. Ta implementacja scramblera
		    posiada wadę konieczności synchronizacji między scramblerem, a descramblerem przez zastosowanie zakodowanego słowa
		    inicjalizującego przeładowanie ciągu znaków losowych do stanu początkowego, dzieje się to na początku każdej nowej ramki
		    danych. Scrambler addytywny jest zarówno descramblerem bez konieczności dodatkowych modyfikacji. 
		    
		    \vspace{0.5cm}
		    \begin{center}
		    	\results[0.7]{sc_add}{Scrambler addytywny\\źródło: wikipedia.org}
		    \end{center}
		    
	    \subparagraph{Multiplikacyjny (samo-synchronizujący)\\}
		    Scramblery multyplikacyjne, nazywane są tak ponieważ wykonują mnożenie sygnału wejściowego z funkcją przejścia ciągu
		    znaków scramblera. W implementacji zgodnej ze standardem V.34 dokonuje operacji xor na 18 i 23 bicie klucza, jest
		    to tzw. funkca przejścia scramblera. Wynikiem jest pierwszy bit ciągu klucza, po operacji xor z bitem wejściowym oraz
		    wynikiem funkcji przejścia. Skrambler multiplikacyjny jest rekurencyjny, a descrambler nie, przez co ich konstrukcje
		    różnią się. Skramblery multiplikacyjne, w przeciwieństwie do addytywnych, nie wymagają synchronizacji ciągu znaków
		    z deskramblerami.
		    
		    \vspace{0.5cm}
		    \begin{center}
		    	\results[0.7]{sc_mul1}{Scrambler multiplikacyjny V.34\\źródło: wikipedia.org}
		    	\vspace{0.5cm}
		    	\results[0.7]{sc_mul2}{Descrambler multiplikacyjny V.34\\źródło: wikipedia.org}
		    \end{center}

\section{Opis symulatora}
    \subsection{Założenia}

    \subsection{Opis programów}

\section{Eksperymenty symulacyjne}
    \subsection{Plan eksperymentów}

    \subsection{Wyniki eksperymentów}

    \subsection{Wnioski}

\section{Literatura}

\newpage
\section{Wykresy}
    \subsection{DVB}
    \begin{minipage}{.5\textwidth}
        \results{DVB_00}{Dane wejściowe: ciąg zer}
        \vspace{1.5cm}
        \results{DVB_10}{Dane wejściowe: ciąg jedynek}
        \vspace{1.5cm}
        \results{DVB_R0}{Dane wejściowe: ciąg losowych bitów}
    \end{minipage}%
    \hspace{1cm}
    \begin{minipage}{.5\textwidth}
        \results{DVB_01}{Dane wejściowe: ciąg zer}
        \vspace{1.5cm}
        \results{DVB_11}{Dane wejściowe: ciąg jedynek}
        \vspace{1.5cm}
        \results{DVB_R1}{Dane wejściowe: ciąg losowych bitów}
    \end{minipage}

\subsection{V34}
    \begin{minipage}{.5\textwidth}
        \results{V34_00}{Dane wejściowe: ciąg zer}
        \vspace{1.5cm}
        \results{V34_10}{Dane wejściowe: ciąg jedynek}
        \vspace{1.5cm}
        \results{V34_R0}{Dane wejściowe: ciąg losowych bitów}
    \end{minipage}%
    \hspace{1cm}
    \begin{minipage}{.5\textwidth}
        \results{V34_01}{Dane wejściowe: ciąg zer}
        \vspace{1.5cm}
        \results{V34_11}{Dane wejściowe: ciąg jedynek}
        \vspace{1.5cm}
        \results{V34_R1}{Dane wejściowe: ciąg losowych bitów}
    \end{minipage}

\subsection{Bit Error Rate}
        \results{BER_DVB}{BER dla DVB}
        \vspace{1.5cm}
        \results{V34_10}{Dane wejściowe: ciąg jedynek}
        \vspace{1.5cm}
        \results{V34_R0}{Dane wejściowe: ciąg losowych bitów}


\end{document}