\documentclass[polish, 11pt]{article}
    
\usepackage[a4paper, margin=25mm]{geometry}
\usepackage{babel,polski}
\usepackage[utf8]{inputenc}
\usepackage[T1]{fontenc}
\usepackage{booktabs,multirow,multicol}

\usepackage{graphicx}
\graphicspath{{DVB/} {V34/}}

\usepackage{xcolor}
\usepackage[font=small,labelfont=bf]{caption}
\captionsetup[figure]{name=Rys.}

\newcommand{\results}[2]{
    \includegraphics[width=\textwidth]{#1}
    \captionof{figure}{#2\label{fig:#1}}
}

\begin{document}
\begin{titlepage}
    \centering
    {\scshape\LARGE Niezawodność i diagnostyka układów cyfrowych 2\\ projekt \par}
    \vspace{1cm}
    {\scshape\Large,,Unikanie utraty synchronizacji przy pomocy randomizacji (Scrambling)''\par}
    \vspace{2cm}
    {\itshape\Large Janusz Długosz --- 235746\/\par}
    {\itshape\Large Jakub Dorda --- 235013\/\par}
    {\itshape\Large Marcin Kotas --- 235098\/\par}
    \vfill
    Prowadzący:\par
    Doc.~dr inż.~Jacek \textsc{Jarnicki}

    \vfill

    {\large Wrocław, \today\par}

\end{titlepage}

\section{Wstęp}

\section{Scramblery}
    \subsection{Po co?}

    \subsection{Typowe rozwiązania}

\section{Opis symulatora}
    \subsection{Założenia}

    \subsection{Opis programów}

\section{Eksperymenty symulacyjne}
    \subsection{Plan eksperymentów}

    \subsection{Wyniki eksperymentów}

    \subsection{Wnioski}

\section{Literatura}

\newpage
\section{Wykresy}
    \subsection{DVB}
    \begin{minipage}{.5\textwidth}
        \results{DVB_00}{Dane wejściowe: ciąg zer}
        \vspace{1.5cm}
        \results{DVB_10}{Dane wejściowe: ciąg jedynek}
        \vspace{1.5cm}
        \results{DVB_R0}{Dane wejściowe: ciąg losowych bitów}
    \end{minipage}%
    \hspace{1cm}
    \begin{minipage}{.5\textwidth}
        \results{DVB_01}{Dane wejściowe: ciąg zer}
        \vspace{1.5cm}
        \results{DVB_11}{Dane wejściowe: ciąg jedynek}
        \vspace{1.5cm}
        \results{DVB_R1}{Dane wejściowe: ciąg losowych bitów}
    \end{minipage}

\subsection{V34}
    \begin{minipage}{.5\textwidth}
        \results{V34_00}{Dane wejściowe: ciąg zer}
        \vspace{1.5cm}
        \results{V34_10}{Dane wejściowe: ciąg jedynek}
        \vspace{1.5cm}
        \results{V34_R0}{Dane wejściowe: ciąg losowych bitów}
    \end{minipage}%
    \hspace{1cm}
    \begin{minipage}{.5\textwidth}
        \results{V34_01}{Dane wejściowe: ciąg zer}
        \vspace{1.5cm}
        \results{V34_11}{Dane wejściowe: ciąg jedynek}
        \vspace{1.5cm}
        \results{V34_R1}{Dane wejściowe: ciąg losowych bitów}
    \end{minipage}

\subsection{Bit Error Rate}
        \results{BER_DVB}{BER dla DVB}
        \vspace{1.5cm}
        \results{V34_10}{Dane wejściowe: ciąg jedynek}
        \vspace{1.5cm}
        \results{V34_R0}{Dane wejściowe: ciąg losowych bitów}


\end{document}