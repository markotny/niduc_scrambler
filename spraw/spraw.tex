\documentclass[polish, 11pt]{article}
    
\usepackage[a4paper, margin=25mm]{geometry}
\usepackage{babel,polski}
\usepackage[utf8]{inputenc}
\usepackage[T1]{fontenc}
\usepackage{booktabs,multirow,multicol}

\usepackage{graphicx}
\graphicspath{{DVB/} {V34/} {img/}}

\usepackage{xcolor}
\usepackage[font=small,labelfont=bf]{caption}
\captionsetup[figure]{name=Rys.}

\newcommand{\results}[3][1.0]{
    \includegraphics[width=#1\textwidth]{#2}
    \captionof{figure}{#3\label{fig:#2}}
}

\begin{document}
\begin{titlepage}
    \centering
    {\scshape\LARGE Niezawodność i diagnostyka układów cyfrowych 2\\ projekt \par}
    \vspace{1cm}
    {\scshape\Large,,Unikanie utraty synchronizacji przy pomocy randomizacji (Scrambling)''\par}
    \vspace{2cm}
    {\itshape\Large Janusz Długosz --- 235746\/\par}
    {\itshape\Large Jakub Dorda --- 235013\/\par}
    {\itshape\Large Marcin Kotas --- 235098\/\par}
    \vfill
    Prowadzący:\par
    Doc.~dr inż.~Jacek \textsc{Jarnicki}

    \vfill

    {\large Wrocław, \today\par}

\end{titlepage}

\tableofcontents
\newpage

\section{Wstęp}

\section{Scramblery - teoria}
    \subsection{Przeznaczenie}
	    Scramblery są kluczowym elementem warstwy fizycznej systemu transmisji bezprzewodowej, poza zapewnieniem
	    kodowania interwałowego oraz modulacji transmisji, powodami dla których używa się układów scaramblujących są:
	    \begin{itemize}
	    	\item Dokładne odtworzenie czasu w układzie odbiorczym potrzebnego do synchronizacji transmisji,
	    	bez konieczności używania redundantnych znaków końca ramki. Rozwiązuje to problem potrzeby stosowania
	    	układu odtwarzającego taktowanie. Automatyzuje kontrolę wzmocnienia sygnału oraz pracę innych układów
	    	przetwarzających sygnał, dzięki eliminacji długich ciągów zer i jedynek.
	    	\item Zmniejszenie zakłóceń sygnału między falami nośnymi, bardziej zróżnicowane dane zapewniają
	    	rozproszenie widma sygnału co przekłada się na mniejszą wrażliwość na zakłócenia między sąsiednimi
	    	kanałami transmisji.
	    	\item Prosta metoda szyfrowania transmisji, stosowana dla sygnałów analogowych. Zarówno scrambler
	    	w urządzeniu nadawczym jak i descrambler w urządzeniu odbiorczym muszą posiadać ten sam klucz/ciąg
	    	znaków w celu poprawnego dekodowania danych.
	    \end{itemize}
	    
    \subsection{Typy scramblerów}
	    \subparagraph{Addytywny (synchroniczny)\\}
		    Scramblery synchroniczne przetwarzają dane wejściowe przez zastosowanie podwójnej operacji xor z dwoma najmłodszymi
		    bitami pseudo losowego ciągu znaków oraz bitem wejścia. Ciąg znaków używanych przez scrambler jest przechowywany w
		    pamięci przeznaczonej jedynie do odczytu, czasami zamiast losowego używany jest też pre-kalkulowany, zoptymalizowany ciąg znaków.
		    Po każdej operacji ciąg - klucz jest przesuwany o jedną pozycję w prawo, a wynik xor dwóch najmłodszych bitów zapisywany na
		    pozycji pierwszej. Ta implementacja scramblera posiada wadę konieczności synchronizacji między scramblerem, a descramblerem
		    przez zastosowanie zakodowanego słowa inicjalizującego przeładowanie ciągu znaków losowych do stanu początkowego, dzieje się
		    to na początku każdej nowej ramki danych. Scrambler addytywny jest zarówno descramblerem bez konieczności dodatkowych modyfikacji. 
		    
		    \vspace{0.5cm}
		    \begin{center}
		    	\results[0.7]{sc_add}{Scrambler addytywny\\źródło: wikipedia.org}
		    \end{center}
		    
	    \subparagraph{Multiplikacyjny (samo-synchronizujący)\\}
		    Scramblery multyplikacyjne, nazywane są tak ponieważ wykonują mnożenie sygnału wejściowego z funkcją przejścia ciągu
		    znaków scramblera. W implementacji zgodnej ze standardem V.34 dokonuje operacji xor na 18 i 23 bicie klucza, jest
		    to tzw. funkca przejścia scramblera. Wynikiem jest pierwszy bit ciągu klucza, po operacji xor z bitem wejściowym oraz
		    wynikiem funkcji przejścia. Scrambler multiplikacyjny jest rekurencyjny, a descrambler nie, przez co ich konstrukcje
		    różnią się. Scramblery multiplikacyjne, w przeciwieństwie do addytywnych, nie wymagają synchronizacji ciągu znaków
		    z descramblerami.
		    
		    \vspace{0.5cm}
		    \begin{center}
		    	\results[0.7]{sc_mul1}{Scrambler multiplikacyjny V.34\\źródło: wikipedia.org}
		    	\vspace{0.5cm}
		    	\results[0.7]{sc_mul2}{Descrambler multiplikacyjny V.34\\źródło: wikipedia.org}
		    \end{center}

\section{Opis symulatora}
    \subsection{Założenia}
     Napisany program miał za zadanie wytworzyć ciąg zer, ciąg jedynek oraz losowy ciąg bitów (wszystkie o długości 1000000). Następnie symulowane było scramblowanie dla każdego z przypadków za pomocą scramblera addytywnego DVB oraz multiplikacyjnego V34. Tak przetworzone dane przesyłane są przez kanał, który dla zadanego prawdopodobieństwa przekłamuje bity w ciągu danych. Po tej operacji wynik jest descramblowany i dla prawdopodobieństwa z przedziału [0, 0.1] liczony jest stopa błędów BER, czyli stosunek ilości przekłamanych do wszystkich bitów.
    \subsection{Opis programów}
    	\subparagraph{Generacja losowego ciągu bitów \\}
    	 Generuje ona najpierw ciąg losowych liczb rzeczywistych z przedziału [0,1]. Liczby były losowane rozkładem równomiernym.
        Następnie liczby powyżej zadanego progu (w czasie eksperymentów próg = 0.5) zamieniane są na bity 1,
        a liczby poniżej progu zamieniane są na 0.
    	\subparagraph{Scrambler DVB\\}
    	Bufor scramblera ma rozmiar 15 bitów. Do funkcji przekazywana jest tablica zawierająca bity  oraz zadane jest słowo synchronizujące \texttt{[ 1 0 0 1 0 1 0 1 0 0 0 0 0 0 0 ]}. Pętla w funkcji wykonuje się tyle razy ile mamy bitów w danych wejściowych. Za każdym razem jest wykonywana operacja xor bitów nr 14 i 15, a następnie na wyniku tego działania jest wykonywana operacja xor z aktualnym bitem danych, wynikiem jest bit wyjściowy. Dodatkowo słowo synchronizujące jest przesuwane o jeden bit w prawo, a jego  bitowi nr 1 przypisywana jest wcześniej obliczony xor bitów nr 14 i 15.
    	\subparagraph{Descrambler DVB\\}
    	Descrambler DVB działa tak samo jak scrambler. Wynika to z tego, że jest addytywny. Aby zdeszyfrować poprawnie dane należy użyć tego samego słowa synchronizującego co w scramblerze.  
    	\subparagraph{Scrambler V34\\}
    
    	Bufor scramblera ma rozmiar 23 bitów. Do funkcji przekazywana jest tablica zawierająca bity. W przeciwieństwie do scramblera DVB nie używa zadanego słowa synchronizującego. Na początku bufor wypełniony jest zerami. Pętla w funkcji wykonuje się tyle razy ile mamy bitów w danych wejściowych.  Za każdą iteracją pętli obliczany jest xor bitów nr 18 i 23. Następnie bufor przesuwany jest w prawo o jeden bit. Bit nr 1 będzie miał wartość działania xor aktualnych danych wejściowych oraz wcześniej obliczonego xor bitów nr 18 i 23. Taką samą wartość będzie miał bit wyjściowy.
    	\subparagraph{Descrambler V34\\}
    	Bufor descramblera ma rozmiar 23 bitów. Do funkcji przekazywana jest tablica zawierająca bity. Bufor na początku wypełniony jest zerami. Pętla w funkcji wykonuje się tyle razy ile mamy bitów w danych wejściowych. W pętli obliczany jest xor 18 i 23 bitu. Następnie bufor przesuwany jest o jeden bit w prawo. Jego bitem nr 1 będzie aktualny bit danych wejściowych. Bitem wynikowym jest xor aktualnego bitu danych wejściowych oraz wcześniej obliczonego xor bitów 18 i 23.
    	\subparagraph{Kanał transmisyjny\\}
    	Do funkcji przekazywana jest tablica bitów oraz prawdopodobieństwo z jakim pojedynczy bit może zostać przekłamany. Dla każdego bitu z danych wejściowych losowana jest liczba rzeczywista z przedziału [0,1], jeśli będzie ona mniejsza od zadanego prawdopodobieństwa to na bicie z danych wejściowych zostanie przeprowadzona negacja, jeśli będzie większa to bit zostanie zostawiony bez zmian.
\section{Eksperymenty symulacyjne}
    \subsection{Plan eksperymentów}
        Eksperymenty przeprowadzone są dla każdego zestawu danych --- ciągi zer, jedynek oraz losowych bitów.
        Rozmiar danych dla którego przedstawiono wykresy wynosił 1000000 elementów.

        W pierwszej kolejności ciągi danych są przekazywane do obu scramblerów.
        Funkcja monitor generuje wykres przedstawiający liczbę wystąpień oraz długości ciągów zer i jedynek w danych.

        W celu wyznaczenia wskaźnika Bit Error Rate dane zescramblowane przepuszczane są przez kanał.
        Kanał z zadanym prawdopodobieństwem zamienia kolejne bity danych na przeciwne.
        W ramach eksperymentu prawdopodobieństwo przekłamania zmieniane jest od 0\% do 10\% co 0,5\%.

        Po wyjściu z kanału dane są descramblowane, a następnie porównywane z oryginalnym zestawem danych.
        Liczba błędnych bitów dzielona jest przez rozmiar danych, aby uzyskać wskaźnik BER.\
        
    \subsection{Wyniki eksperymentów}
        Wykresy~\ref{fig:DVB_00},~\ref{fig:DVB_01} oraz~\ref{fig:V34_00},~\ref{fig:V34_01} przedstawiają ciąg zer po scramblingu odpowiednio DVB oraz V34.
        W przypadku scramblera V34 żadne bity nie zostały zamienione, ponieważ na początku w buforze scramblera znajdują się same zera.
        Operacja \(0 \oplus 0\) zawsze da w wyniku 0, więc stan bufora nigdy się nie zmieni.
        Scrambler DVB zawsze zaczyna od słowa SYNC w buforze, co eliminuje ten problem.

        W przypadku ciągu jedynek (wykresy~\ref{fig:DVB_10},~\ref{fig:DVB_11} oraz~\ref{fig:V34_10},~\ref{fig:V34_11}) oba scramblery zachowują się już poprawnie.
        Liczba wystąpień poszczególnych długości ciągów jest podobna dla obu rodzajów scramblera. 
        Scrambler V34 wygenerował jednak najdłuższy ciąg długości 18, natomiast DVB długości 14.

        Dla danych generowanych losowo (wykresy~\ref{fig:DVB_R0},~\ref{fig:DVB_R1} oraz~\ref{fig:V34_R0},~\ref{fig:V34_R1})
        oba scramblery dały podobne wyniki.

        Bit Error Rate scramblera V34 (wykres~\ref{fig:BER_V34}) w zmierzonym zakresie jest znacznie gorszy od scramblera DVB (wykres~\ref{fig:BER_DVB}).
        Rośnie on około 2 razy szybciej.

    \subsection{Wnioski}
        Scramblery działają poprawnie, dane wyjściowe składają się głównie z krótkich ciągów tych samych bitów.
        Bieżąca implementacja kanału nie jest wystarczająco dopracowana na potrzeby symulacji.
        Przekłamania bitów powinny odbywać się z prawdopodobieństwem, które rośnie wraz z długością ciągu tych samych bitów w danych.
        Dopiero wtedy widoczne stałyby się zyski wynikające z użycia scramblerów.
        
        Obecnie z eksperymentu wynika, iż użycie scramblera skutkuje w najlepszym wypadku niepogorszeniem współczynnika BER (dla DVB BER = prawdopodobieństwu przekłamania),
        a w najgorszym przypadku dwukrotnym wzroście współczynnika BER (dla V34).
        Wzrost BER dla V34 wynika z faktu, że bity danych ładowane są do bufora, a więc przekłamany bit może skutkować późniejszym dodatkowym przekłamaniem.

        Eksperyment można więc uznać za nieudany, ponieważ użyty symulator nie uwzględnia istotnych zjawisk zachodzących podczas transmisji.

\newpage
\section{Wykresy}
    \subsection{DVB}

    \begin{minipage}[t]{.5\textwidth}
        \results{DVB_00}{Dane wejściowe: ciąg zer}
    \end{minipage}%
    \hspace{1cm}
    \begin{minipage}[t]{.5\textwidth}
        \results{DVB_01}{Dane wejściowe: ciąg zer}
    \end{minipage}%
    \vspace{1.5cm}
    \begin{minipage}[t]{.5\textwidth}
        \results{DVB_10}{Dane wejściowe: ciąg jedynek}
    \end{minipage}%
    \hspace{1cm}
    \begin{minipage}[t]{.5\textwidth}
        \results{DVB_11}{Dane wejściowe: ciąg jedynek}
    \end{minipage}%
    \vspace{1.5cm}
    \begin{minipage}[t]{.5\textwidth}
        \results{DVB_R0}{Dane wejściowe: ciąg losowych bitów}
    \end{minipage}%
    \hspace{1cm}
    \begin{minipage}[t]{.5\textwidth}
        \results{DVB_R1}{Dane wejściowe: ciąg losowych bitów}
    \end{minipage}

\subsection{V34}
    \begin{minipage}[t]{.5\textwidth}
        \results{V34_00}{Dane wejściowe: ciąg zer}
    \end{minipage}%
    \hspace{1cm}
    \begin{minipage}[t]{.5\textwidth}
        \results{V34_01}{Dane wejściowe: ciąg zer}
    \end{minipage}%
    \vspace{1.5cm}
    \begin{minipage}[t]{.5\textwidth}
        \results{V34_10}{Dane wejściowe: ciąg jedynek}
    \end{minipage}%
    \hspace{1cm}
    \begin{minipage}[t]{.5\textwidth}
        \results{V34_11}{Dane wejściowe: ciąg jedynek}
    \end{minipage}%
    \vspace{1.5cm}
    \begin{minipage}[t]{.5\textwidth}
        \results{V34_R0}{Dane wejściowe: ciąg losowych bitów}
    \end{minipage}%
    \hspace{1cm}
    \begin{minipage}[t]{.5\textwidth}
        \results{V34_R1}{Dane wejściowe: ciąg losowych bitów}
    \end{minipage}

\subsection{Bit Error Rate}
        \results{BER_DVB}{BER dla DVB}
        \vspace{1.5cm}
        \results{BER_V34}{BER dla V34}

\end{document}